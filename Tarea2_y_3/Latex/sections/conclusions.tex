Basado en los resultados obtenidos, se observa que para cadenas vacías o cadenas totalmente idénticas, el algoritmo de fuerza bruta presenta tiempos de ejecución más bajos. Esto se debe a que, en estos casos, las recursiones terminan rápidamente al cumplirse siempre alguno de los casos base, lo que reduce significativamente el número de operaciones realizadas y genera un comportamiento cercano a constante. Por otro lado, el algoritmo de programación dinámica, independientemente de la entrada, debe recorrer y llenar completamente la matriz de soluciones parciales, lo que implica un mayor tiempo de ejecución en comparación con la fuerza bruta, incluso en casos simples.
En los casos en los que las cadenas no cumplen los casos base del algoritmo de fuerza bruta, se observa una tendencia notable en los tiempos de ejecución. A medida que aumenta el tamaño y la complejidad de las cadenas de entrada, el tiempo de ejecución del algoritmo de fuerza bruta crece exponencialmente, debido a la naturaleza combinatoria de las llamadas recursivas necesarias para evaluar todas las posibles soluciones. Este comportamiento es consistente con la complejidad temporal teórica previamente calculada para este algoritmo, lo que lo hace impráctico para entradas de tamaño considerable.

Por el contrario, el paradigma de programación dinámica muestra un crecimiento en los tiempos de ejecución mucho más controlado, siguiendo una tendencia lineal respecto al número de subproblemas resueltos. Este crecimiento más lento se alinea con la complejidad temporal cuadrática esperada, derivada de la construcción y evaluación de la matriz de soluciones parciales. Esto confirma la superioridad del enfoque de programación dinámica en términos de escalabilidad y eficiencia computacional cuando se trata de entradas grandes o complejas.