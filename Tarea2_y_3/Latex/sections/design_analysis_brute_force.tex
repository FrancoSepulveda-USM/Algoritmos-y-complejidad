\subsubsection{Descripción solución diseñada}
Para implementar la distancia mínima entre dos cadenas utilizando fuerza bruta se pasan como parametros la cadena 1(s1), la cadena 2(s2), n(tamaño s1) y m(tamaño s2), en este algoritmo ambas cadenas se recorreran recursivamente desde sus caracteres finales hasta sus caracteres iniciales. \\
Primero se identificaron 3 casos base, cuando la cadena 1 es vacía, en este caso el costo total corresponderá al costo de insertar todas las letras de la cadena 2.
Cuando la cadena 2 es vacía, aquí el costo total corresponderá al costo de eliminar todas las letras de la cadena 1.
Y el ultimo caso es cuando las dos letras apuntadas en ambas cadenas son iguales, en este caso se procede a llamar recursivamente a la función para acceder a la siguiente letra de la cadena 1 y de la cadena 2.\\
Luego de los casos bases, se procede a llamar recursivamente a la función con el fin de calcular los costo de insertar, eliminar, sustituir y transponer, quedandose con el minimo entre estos. 
Al ser de fuerza bruta, este algoritmo calculará todos los costos de todas las combinaciones posibles, de esta forma se irá quedando cada vez con el costo menor entre los 4 calculados. Y así seguirá hasta quedarse finalmente con el costo mínimo de todas las recursiones hechas.
\subsubsection{Análisis temporal y espacial}
n: Tamaño cadena 1.
m: Tamaño cadena 2.

Análisis temporal: El peor caso ocurre cuando los caracteres del mismo indice entre las dos cadenas no son iguales y cuando se cumplen las condiciones para transponer, al cumplirse esto, se hacen siempre las 4 recursiones. Esto genera un crecimiento exponencial en el numero total de operaciones ya que cada subproblema genera hasta 4 nuevos subproblemas, formando un arbol de recursión aproximadamente de altura min(n,m). Por lo tanto, la complejidad en el peor caso es:
\[O(4^{max(n,m)})\]
Analisis espacial: El programa al no usar estructuras de datos adicionales, el análisis espacial solo se hace sobre el stack, en el peor caso, el algoritmo realiza como máximo n + m recursiones, ya que en cada llamada recursiva al menos uno de los indices n o m disminuye. Esto asegura que la profundidad máxima de la pila estará acotada por la suma de los tamaños de las cadenas(n y m). La complejidad espacial queda:
\[O(n+m)\]
\subsubsection{Algoritmo utilizando fuerza bruta}
\begin{algorithm}[H]
    \SetKwProg{myproc}{Procedure}{}{}
    \SetKwFunction{distanciaFB}{distanciaFB}  % Cambia 'AlgorithmName' por el nombre del enfoque elegido
    
    \DontPrintSemicolon
    \footnotesize

    % Definición del algoritmo principal
    \myproc{\distanciaFB{s1, s2, n, m}}{
    \uIf{n = 0}{
    	$costo \leftarrow 0$\;
    	\For{$i \leftarrow 0$ \KwTo $m$}{
    		$costo \leftarrow costo + costo\_ins(s2[i])$\;
    	}
        \Return $costo$;  
    }
    \uIf{m = 0}{
    	$costo \leftarrow 0$\;
    	\For{$i \leftarrow 0$ \KwTo $n$}{
    		$costo \leftarrow costo + costo\_del(s1[i])$\;
    	}
    	\Return $costo$; 
    }
    
    
    \uIf{s1[n-1] = s2[m-1]}{
    	\Return \distanciaFB{s1,s2,n-1,m-1}; 
    }
    
    
    $costo\_insertar = \distanciaFB(s1,s2,n,m-1) + costo\_ins(s2[m-1])$;
    
    $costo\_delete = \distanciaFB(s1,s2,n-1,m) + costo\_del(s1[n-1])$;
    
    $costo\_sustituir  = \distanciaFB(s1,s2,n-1,m-1) + costo\_sub(s1[n-1],s2[m-1])$;
    
    $costo\_transponer = INT\_MAX$;
    
    \uIf{($s1[n-1] = s2[m-2]$ \textbf{and} $s1[n-2] = s2[m-1]$ \textbf{and} $n > 1$ \textbf{and} $m > 1$)}{
    	$costo\_transponer \gets distanciaFB(s1, s2, n-2, m-2) + costo\_trans(S1[n-1], S1[n-2])$\;
    }
    
    
    \Return $\min(costo\_insertar,costo\_delete,costo\_sustituir,costo\_transponer)$;
    
    }
    \caption{Algoritmo de fuerza bruta para encontrar la distancia.}
    \label{alg:mi_algoritmo_1}
\end{algorithm}

\subsubsection{Ejemplo paso a paso}
Dadas las cadenas: s1 = $"$ab$"$,  s2 = $"$ba$"$. Si los costos de las operaciones son:
Inserción : 1, Eliminación: 1, Sustitución : 1, Transposición : 1


\subsection*{Llamada inicial}
La llamada inicial es: distanciaFB($"$ab$"$, $"$ba$"$, 2, 2).
Evaluación paso a paso

1. Caso general: \( s1[1] \neq s2[1] \).\\
   Evaluamos los costos de cada operación: Costo insertar: distanciaFB($"$ab$"$, $"$ba$"$, 2, 1) + 1, Costo eliminar: distanciaFB($"$ab$"$, $"$ba$"$, 1, 2) + 1, Costo sustituir: distanciaFB($"$ab$"$, $"$ba$"$, 1, 1) + 1, Costo transponer: distanciaFB($"$ab$"$, $"$ba$"$, 0, 0) + 1 (condición cumplida).

2. Subproblema: distanciaFB($"$ab$"$, $"$ba$"$, 1, 1): \( s1[0] \neq s2[0] \), evaluamos los costos: Costo insertar: distanciaFB($"$ab$"$, $"$ba$"$, 1, 0) + 1 = 2, Costo eliminar: distanciaFB($"$ab$"$, $"$ba$"$, 0, 1) + 1 = 2, Costo sustituir: distanciaFB($"$ab$"$, $"$ba$"$, 0, 0) + 1 = 1.
  El mínimo es \( 1 \), correspondiente a la sustitución.\\

3. Subproblema  distanciaFB($"$ab$"$, $"$ba$"$, 1, 2):
   - Evaluamos los costos:
   Costo insertar: distanciaFB($"$ab$"$, $"$ba$"$, 1, 1) + 1 = 2, 
   Costo eliminar: distanciaFB($"$ab$"$, $"$ba$"$, 0, 2) + 1 = 3.
   
   El mínimo es \( 2 \).\\

4. Subproblema distanciaFB($"$ab$"$, $"$ba$"$, 2, 1):**
   - Evaluamos los costos:
   
   Costo eliminar: distanciaFB($"$ab$"$, $"$ba$"$, 1, 1) + 1 = 2.
   
   El mínimo es \( 2 \).\\

5. Resolviendo distanciaFB($"$ab$"$, $"$ba$"$, 2, 2):

   Costo transponer: distanciaFB($"$ab$"$, $"$ba$"$, 0, 0) + 1 = 1.

   El mínimo global es \( 1 \), correspondiente a la transposición.

\subsection*{Resultado}
Finalmente el costo mínimo para transformar \( s1 \) en \( s2 \) es:
\[
\text{distanciaFB($"$ab$"$, $"$ba$"$, 2, 2)} = 1.
\]

\subsubsection{Impacto de incluir operación de transposición y costos variables en la complejidad}
La inclusión de la operación de transposicion genera que se haga una recusión más, lo que causa que la complejidad original \(O(3^{min(n,m)})\) pase a \(O(4^{min(n,m)})\), esto provoca un crecimiento exponencial aún más rapido. 
Lo que puede provocar que el algoritmo sea impráctico para cadenas de mayor tamaño.
La complejidad espacial sigue siendo de \(O(n+m)\) ya que cada llamada recursiva sigue disminuyendo al menos uno de los indices n o m.

Por otro lado, los costos variables aportan flexibilidad para modelar problemas más realistas, pero no afectan directamente la estructura del arbol de recursión. Sin embargo, puede incrementar la carga computacional en cada nivel debido a que para cada costo debe acceder a las tablas de costos, y aunque esta operación sea de O(1), la acumulación de estos acceso en la ejecución completa del algoritmo puede incrementar el tiempo de cálculo.

En conclusión, aumenta la complejidad temporal de \(O(3^{max(n,m)})\)  a \(O(4^{max(n,m)})\) y la complejidad espacial se mantiene \(O(n+m)\).