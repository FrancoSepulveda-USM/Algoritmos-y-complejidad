La distancia mínima entre dos cadenas surge como solución a el problema de corrección de errores en códigos de información, en particular, a la necesidad de medir la diferencia entre dos cadenas de texto o secuencias, determinando el numero mínimo de transformaciones 
necesarias para convertir una cadena en otra. 
Esta solución, conocida como "La distancia de Levenshtein" pertenece al campo de Análisis y Diseño de algoritmos en Ciencias de la Computación, un área crucial en la resolución de problemas complejos con impacto significativo tanto en aplicaciones prácticas como en la investigación académica.\\

Tal vez muchas personas, yo incluido, no tenian en cuenta la relevancia  que puede tener la solucion a problemas de correccion de errores, pero esto se puede ver reflejado en la aplicabilidad en ámbitos de la vida real que tiene este problema, tales como:

-Correccion ortográfica: Consiste en sugerir la palabra correcta cuando un usuario 
comete errores tipográficos.

-Motores de búsqueda: Ayuda a mejorar los resultados de búsqueda cuando hay errores
tipográficos o variaciones en las palabras.

-Biología computacional: Aplicado en secuencias de ADN, ARN o proteínas.
En este caso se mide la "distancia" entre secuencias genéticas para detectar mutaciones
o comparar diferentes organismos.

-Comparación de textos para análisis de plagio: Compara textos y detecta
similitudes o plagio entre documentos.

-Machine learning y natural languaje processing: La distancia se usa como una métrica
de similitud para entrenar modelos de aprendizaje automático en el campo de procesamiento
de lenguaje natural(NLP). Útil en tareas como traducción automática,
reconocimiento de entidades nombradas,etc.\\

En este trabajo se aplicará la distancia de Levenshtein pero con ciertas modificaciones,
agregando la operación de transposición y variando los costos de las operaciones.
¿Como afecta la inclusión de transposición y costos variables? Este informa busca 
responder a esta cuestión mediante el análisis de dos enfoques: un algoritmo basado en
programación dinámica y otro en fuerza bruta, aplicados a este problema extendido.

Se compararan ambas implementaciones en términos de estructura, tiempos de ejecución y uso de memoria al procesar cadenas con distintas características.
Esta comparación entre paradigmas distintos permite comprender la relación entre la complejidad teorica y los resultados experimentales, parte esencial para un curso de pregrado. 

