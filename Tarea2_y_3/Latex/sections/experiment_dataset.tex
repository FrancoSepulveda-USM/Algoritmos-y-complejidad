Se crearon seis datasets en archivos txt. La creación si hizo mediante el uso de Python, en el programa se generan cadenas de distintas característcas y se escriben en archivos txt. El formato en que se escriben los datos es el siguiente:
La primera linea corresponde a la longitud de las dos cadenas, separadas por un espacio, en la siguiente linea se imprime la cadenas 1 y en la siguiente linea se imprimer la cadena 2, luego se hace una linea vacía y se procede a imprimir las siguientes dos cadenas con sus respectivos largos en el formato antes mencionado.
Aclarar que las cadenas se crean de forma aleatoria.\\
\begin{itemize}
    \item \textbf{Descripción de los datasets}
    \begin{itemize}
        \item \textbf{datasets1.txt}: Contiene cadenas vacías, ambas pueden ser vacías o solo una. De diferentes tamaños.
        \item \textbf{datasets2.txt}: Contiene cadenas idénticas de mismo tamaño. El tamaño varía.  
        \item \textbf{datasets3.txt}: Contiene cadenas de distinto tamaño que poseen caracteres repetidos.
        \item \textbf{datasets4.txt}: Contiene cadenas distintas de igual tamaño.
        \item \textbf{datasets5.txt}: Contiene cadenas distintas de diferente tamaño.
        \item \textbf{datasets6.txt}: Contiene cadenas que tienen varias transposiciones, hay cadenas de igual tamaño y otras de distinto tamaño.
    \end{itemize}
\end{itemize} 

El código principal se encarga de leer las cadenas provenientes de los datasets y calcular la mínima distancia entre ellas, calculando también el tiempo de ejecución de los dos algoritmos.
Para la experimentación las cadenas tendran un tamaño máximo de 14 caracteres, pasado este tiempo el algoritmo de fuerza bruta demora mucho debido a que hace muchas combinaciones.