\subsubsection{Descripción de la solución recursiva}
Para implementar la solución mediante programación dinámica se utilizó el enfoque bottom up. Primero se crea una tabla de tamaño $(n+1) \times (m+1)$, que representa los subproblemas para medir la distancia entre prefijos de las cadenas s1 y s2. Luego esta se inicializa en la función \texttt{inicializarMatriz}. Inicialización: 

-tabla[0][0] = 0, que corresponde al caso en que ambas cadenas están vacías.

-Primera fila(n=0): Representa cuando la cadena s1 está vacía, por lo que el costo acumulado en cada celda es la suma del costo de insertar todos los caracteres de s2 hasta la posición correspondiente:\\
\[tabla[0][j] = \sum_{j=1}^{m} costo\_ins(s2[j-1])\]

- Primera columna: llenar la primera columna de la tabla(m=0) implica que la cadena 2 es vacía, por lo que el costo acumulado es la suma del costo de eliminar todos los caracteres de s1 hasta la posición correspondiente:\\
\[tabla[i][0] = \sum_{i=1}^{n} costo\_del(s1[i-1])\]
Una vez inicializada la matriz, el siguiente paso es llenar las demás casillas de la tabla, para esto se recorre la tabla iterativamente considerando los costos de cada operación y escogiendo el mínimo entre ellos. De esta forma cada casilla tendrá el costo mínimo entre los 4 costos calculados(costo inserción, costo eliminar, costo sustituir, costo transponer).
\[
\text{tabla}[i][j] = \min\left(
\begin{aligned}
    &\text{tabla}[i][j-1] + \text{costo\_ins}(s2[j-1]), \\
    &\text{tabla}[i-1][j] + \text{costo\_del}(s1[i-1]), \\
    &\text{tabla}[i-1][j-1] + 
    \begin{cases} 
        0 & \text{si } s1[i-1] = s2[j-1] \\
        \text{costo\_sub}(s1[i-1], s2[j-1]) & \text{en otro caso}
    \end{cases}, \\
    &\text{tabla}[i-2][j-2] + \text{costo\_trans}(s1[i-1], s1[i-2]) \text{ (si aplica)}
\end{aligned}
\right)
\]

Una vez completada la tabla, el valor de tabla[n][m] representa la mínima distancia entre las cadenas s1 y s2. Este enfoque garantiza que todos los subproblemas sean resueltos eficientemente, reutilizando soluciones previamente calculadas y evitando recalcular subproblemas redundantes.
\subsubsection{Relación de recurrencia}
$D(n,m) \leftarrow \begin{cases}
  				\max(\sum_{i=1}^{n} costo\_del(s1[i-1]), \sum_{i=1}^{m} costo\_ins(s2[i-1])), & \text{si } \min(n,m) = 0 \\
  				\min\begin{cases}
                        D(n,m-1) + costo\_ins(s2[m-1]),\\
                        D(n-1,m) + costo\_del(s1[n-1]),\\
                        D(n-1,m-1) + costo\_sub(s1[n-1],s2[m-1]),\\
                        D(n-2,m-2) + costo\_trans(s1[n-1],s1[n-2])\\
                    \end{cases}, & \text{caso contrario}
  			\end{cases}$
\subsubsection{Identificación de subproblemas}
Cada subproblema puede representarse como D(i,j) donde i es el tamaño parcial de la cadena  s1 (0 $\leq$ i $\leq$ n) y j es el tamaño parcial de la cadena s2 (0 $\leq$ j $\leq$ m).
D(i,j) calcula el costo mínimo de transformar los primeros i caracteres de s1 en los primeros j caracteres de s2, considerando los costos de las cuatros operaciones existentes: \\
- Inserción: Agrega caracteres a s1 para que coincida con s2\\
- Eliminación: Eliminar caracteres de s1 para que coincida con s2\\ 
- Sustitución: Cambia un carácter de s1 por el correspondiente carácter de s2\\
- Transposición: Intercambiar dos caracteres adyacentes de s1, si es que se cumplen las condiciones.\\
Casos base: \\
- D(0,j): Costo de transformar la cadena 1(s1) vacía en los primeros j caracteres de s2. Corresponde a la suma de los costos de insertar esos j caracteres en s1.\\
- D(i,0): Costo de transformar los primeros i caracteres de la cadena 1 (s1) en una cadena vacía (s2). Corresponde a la suma de los costos de eliminar esos i caracteres de s1.

Para calcular D(i,j) se utilizan los resultados previamente calculados de los subproblemas:\\
- D(i-1,j): Subproblema donde se elimina un carácter de s1.\\
- D(i,j-1): Subproblema donde se inserta un carácter en s1.\\
- D(i-1,j-1): Subproblema donde se susituye un carácter de s1.\\
- D(i-2,j-2): Subproblema donde se realiza una transposición de dos caracteres adyacentes de s1.\\
En resumen, los subproblemas son todas las combinaciones D(i,j) para 0 $\leq$ i $\leq$ n y 0 $\leq$ j $\leq$ m (se incluye el 0 para los casos base). Estos subproblemas permiten dividir el problema principal en partes más pequeñas que se resuelven utilizando la tabla de programación dinámica.
\subsubsection{Estructura de datos y orden de cálculo}
Estructura de datos: Al ser un algoritmo de programación dinámica se necesita una tabla donde  ir guardando los valores de los subproblemas y a partir de estos poder calcular los siguientes. En este algoritmo se utiliza una matriz de dimensiones $(n+1) \times (m+1)$, donde n y m corresponden a los tamaños de la cadena 1(s1) y la cadena 2(s2), respectivamente.
Cada celda de la matriz $tabla[i][j]$ representa el costo mínimo para transformar el prefijo de tamaño i de la cadena s1 en el prefijo de tamaño j de la cadena s2.\\

Orden de cálculo: Como se mencionó anteriormente, el llenado de la matriz sigue un enfoque $"bottom-up"$. 
Primero se llenan los casos base de la matriz, los cuales corresponden a la primera fila y la primera columna.
Luego se recorre la matriz iterativamente mediante dos for. Siguiendo el siguiente recorrido:
tabla[1][1], tabla[1][2], tabla[1,3] hasta tabla[1][m], luego pasa a la siguiente fila. Esto corresponde a un recorrido de izquierda a derecha y de arriba hacia abajo. Para cada celda (i,j) se calcula el costo mínimo considerando las cuatro operaciones posibles.
Al finalizar el cálculo, el valor en tabla[n][m] almacenará la distancia mínima requerida para transformar s1 en s2.

\subsubsection{Análisis temporal y espacial}
Análisis temporal: El código como tal trata sobre recorrer iterativamente una matriz y en cada iteración calcular los costos, lo cual es trabajo constante O(1). En base a esto, el orden de complejidad temporal para este código esta dado por los dos for: O(n*m)

Análisis espacial: Por otro lado el espacio utilizado por el código corresponde a la generación de la matriz de tamaño n+1 * m+1 por lo que la memoria ocupada es O(n*m), las otras variables son O(1).
Complejidad espacial final: O(n*m)

\subsubsection{Algoritmo utilizando programación dinámica}

\begin{algorithm}[H]

    \SetKwProg{myproc}{Procedure}{}{}
    \SetKwFunction{distanciaPD}{distanciaPD}  % Cambia 'AlgorithmName' por el nombre del enfoque elegido
    \SetKwFunction{inicializarMatriz}{inicializarMatriz}  % Función auxiliar de ejemplo
    
    \DontPrintSemicolon
    \footnotesize

    % Definición del algoritmo principal
    \myproc{\distanciaPD{s1, s2, n, m}}{
  	$matriz \leftarrow \inicializarMatriz(s1, s2, n, m)$\;
  	\For{$i \leftarrow 1$ \KwTo $n+1$}{
  		\For{$j \leftarrow 1$ \KwTo $m+1$}{
  			$costo\_insertar = matriz[i][j-1] + costo\_ins(s2[j-1])$;
  			
  			$costo\_eliminar = matriz[i-1][j] + costo\_del(s1[i-1])$;
  			
  			$costo\_sustituir \leftarrow \begin{cases}
  				matriz[i-1][j-1] + 0, & \text{si } s1[i-1] = s2[j-1] \\
  				matriz[i-1][j-1] + costo\_sub(s1[i-1], s2[j-1]), & \text{caso contrario}
  			\end{cases}$
  			
  			$costo\_transponer = INT\_MAX$;  % Inicialización en 0
  		
  			\If{$i \geq 2$ \textbf{and} $j \geq 2$ \textbf{and} $s1[i-1] = s2[j-2]$ \textbf{and} $s1[i-2] = s2[j-1]$}{
  				$costo\_transponer = matriz[i-2][j-2] + costo\_trans(s1[i-1], s1[i-2])$\;
  			}
  			
  			$matriz[i][j] \leftarrow \min(costo\_insertar, costo\_eliminar, costo\_sustituir, costo\_transponer)$;
  		}
  	}
  	\Return $matriz[n][m]$; 
    }
    % Definición de la función auxiliar
    \myproc{\inicializarMatriz{s1, s2, n, m}}{
    % Acciones o cálculos auxiliares
    \STATE Inicializar $matriz$ de tamaño $(n+1) \times (m+1)$;
    
    \STATE $matriz[0][0] \gets 0$;
    
    \For{$i \leftarrow 1$ \KwTo $n+1$}{
    	$matriz[i][0] = matriz[i-1][0] + costo\_del(s1[i-1])$;
    }
    \For{$i \leftarrow 1$ \KwTo $m+1$}{
    	$matriz[0][i] = matriz[0][i-1] + costo\_ins(s2[i-1])$;
    }
    \Return $matriz$;
    }
    \caption{Algoritmo de programación dinámica para encontrar la distancia.}
    \label{alg:mi_algoritmo_2}
\end{algorithm}



\subsubsection{Ejemplo paso a paso}
Dadas las cadenas: s1 = $"$abc$"$ y s2 = $"$acb$"$
y costos de operación:
\[
\text{Costo de inserción} = 1, \quad \text{Costo de eliminación} = 1, \quad \text{Costo de sustitución} = 1, \quad \text{Costo de transposición} = 1.
\]
Paso 1: Inicialización de la tabla

La tabla dinámica tiene dimensiones \( 4 \times 4 \) (considerando índices desde 0). Inicializamos las filas y columnas con los costos acumulados de inserciones y eliminaciones.

\[
\begin{array}{c|c|c|c|c}
    &   & a & c & b \\
    \hline
    & 0 & 1 & 2 & 3 \\
    \hline
    a & 1 &   &   &   \\
    \hline
    b & 2 &   &   &   \\
    \hline
    c & 3 &   &   &   \\
\end{array}
\]

Paso 2: Llenado de la tabla
Para cada celda \( \text{tabla}[i][j] \), calculamos el valor mínimo considerando:

- **Inserción**: \( \text{tabla}[i][j-1] + 1 \),

- **Eliminación**: \( \text{tabla}[i-1][j] + 1 \),

- **Sustitución**: \( \text{tabla}[i-1][j-1] + (0 \text{ si } s1[i-1] = s2[j-1], \text{ 1 en caso contrario}) \),

- **Transposición** (si aplica): \( \text{tabla}[i-2][j-2] + 1 \).

Llenamos la tabla paso a paso:

Iteraciones:
1. \( \text{tabla}[1][1] = \min(1 + 1, 1 + 1, 0 + 0) = 0 \),  
2. \( \text{tabla}[1][2] = \min(0 + 1, 2 + 1, 1 + 1) = 1 \),  
3. \( \text{tabla}[1][3] = \min(1 + 1, 3 + 1, 2 + 1) = 2 \),  
4. \( \text{tabla}[2][1] = \min(2 + 1, 0 + 1, 1 + 1) = 1 \), 
5. \( \text{tabla}[2][2] = \min(1 + 1, 1 + 1, 0 + 1) = 1 \),  
6. \( \text{tabla}[2][3] = \min(1 + 1, 2 + 1, 1 + 1) = 2 \),  
7. \( \text{tabla}[3][1] = \min(3 + 1, 1 + 1, 2 + 1) = 2 \),  
8. \( \text{tabla}[3][2] = \min(2 + 1, 1 + 1, 1 + 0) = 1 \),  
9. \( \text{tabla}[3][3] = \min(1 + 1, 2 + 1, 2 + 1, 0 + 1) = 1 \).

Resultado:
El costo mínimo para transformar \( s1 \) en \( s2 \) es \( \text{tabla}[3][3] = 1 \).


\subsubsection{Impacto de incluir operación de transposición y costos variables en la complejidad}
Inclusión de operación de transposición: El incluir la operación de transposición genera que para calcular el valor de cada celda ahora se considere un cuarto caso, en el que se debe verificar si se cumple la condición de transponer y aparte calcular el costo de realizar esta operación. Este cálculo como tal es trabajo constante adicional, por lo que no afecta a la complejidad temporal y esta sigue siendo $O(n*m)$. Sin embargo, el tiempo total de ejecución puede aumentar ligeramente debido a este cálculo adicional.\\
Costos variables: Aunque los costos sean variables, el costo de cada operación sigue siendo constante O(1). Por tanto, que los costos sean variables no afectan el orden de la complejidad temporal. Por otro lado, para tener los costos de cada operación se crearon tablas globales que contienen los valores correspondientes a cada operación. Esto implica un uso espacio adicional, aunque el impacto es mínimo debido al tamaño fijo de las tablas.