Esta tarea aborda el problema de calcular la distancia mínima de edición entre dos cadenas de texto, utilizando algoritmos de Programación Dinámica y Fuerza Bruta. El objetivo es calcular esta distancia mediante operaciones de inserción, eliminación, sustitución y transposición, cada una con un costo asociado.
El informe incluye la contextualización del problema y su relevancia, presenta los algoritmos junto con ejemplos de su aplicación y análisis de su complejidad(temporal y espacial),explica la estructura y funcionamiento de los programas desarrollados, mide los tiempos de ejecución y el consumo de memoria sometiendo los algoritmos a cadenas de distintas características y finalmente interpreta los resultados.

La importancia de este problema radica en mejorar la eficiencia en tareas como la corrección ortográfica, el análisis de texto y la biología computacional. Por lo mismo
, este estudio puede ofrecer una evaluación empírica que puede guiar futuras investigaciones y optimizaciones en este ámbito.



