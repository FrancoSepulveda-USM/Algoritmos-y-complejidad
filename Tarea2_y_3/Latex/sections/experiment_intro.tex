La experimentación se llevó a cabo en el siguiente entorno de hardware:
\begin{itemize}
    \item \textbf{Procesador}: Intel Core i5-1240P (12ª generación)
    \begin{itemize}
        \item \textbf{Velocidad base}: 1.70 GHz
        \item \textbf{Núcleos}: 12 (6 de rendimiento y 6 de eficiencia)
        \item \textbf{Procesadores lógicos}: 16 (debido a la tecnología Hyper-Threading)
        \item \textbf{Caché}:
        \begin{itemize}
            \item L1: 1.1 MB
            \item L2: 9.0 MB
            \item L3: 12.0 MB
        \end{itemize}
        \item \textbf{Virtualización}: Habilitada
    \end{itemize}
    
    \item \textbf{Memoria RAM}:
    \begin{itemize}
        \item \textbf{Capacidad total}: 8.0 GB
        \item \textbf{Velocidad de memoria}: 4267 MT/s
        \item \textbf{Memoria reservada para hardware}: 301 MB
    \end{itemize}
    
    \item \textbf{Almacenamiento}:
    \begin{itemize}
        \item \textbf{Disco SSD}: Samsung MZVLQ512HBLU-00B
        \item \textbf{Capacidad total del disco}: 477 GB
        \item \textbf{Disco del sistema}: Sí
        \item \textbf{Archivo de paginación}: Sí
    \end{itemize}    
\end{itemize}

Entorno de software: 
\begin{itemize}
    \item \textbf{Sistema operativo}: Ubuntu 22.04.4 LTS
    \item \textbf{Versión compilador g++}: 13.1.0
    \item \textbf{Versión de python}: 13.10.12
    \item \textbf{Versión de valgrind}: 3.18.1
    \item \textbf{Libreria chrono}: Utilizado para calcular el tiempo de ejecución.
    \item \textbf{Libreria sys/resource.h}: Utilizada para calcular el uso de memoria de los algoritmos.
\end{itemize}
    